\section{Introduction}
The Virtual Application Profiler (VAPP) was designed to give programmers
greater understanding and control of their code in response to
the rising complexity in architecture over the past decades.
In this context, architecture complexity refers to both the
complicated hardware units of a single core (such as the Pentium 4)
as well as the new challenges brought forth by the rise of
multicore machines.

Specifically, our goals are to (1) allow programmers to quickly
simulate application execution over a large set of architecture
configurations, (2) assist in the testing and debugging of
programs that run on parallel architecture, and (3) do so in a
transparent manner that allows programmers to easily access and
query those aspects of an application's behavior that are relevant to
these tasks.  To achieve these goals, we have pursued a
log-and-replay approach, in which a trace of application's execution
is stored and subsequently replayed on simulated architecture or
used for other high level analysis.  The logged data is stored
in a manner that allows the programmer to directly inspect or
make custom inquiries over the trace.

The third goal is the primary novel technical contribution of this
work that distinguishes our approach from the multitude of
related profilers and replay tools.  TODO: describe how our
project differs from MemSpy \cite{martonosi1992memspy},
SIGMA \cite{derose2002sigma}, BugNet \cite{narayansamy2005bugnet},
Eraser \cite{savage1997eraser}, and FDR \cite{xu2003fdr}.

We have released VAPP as an open source project and it is available
for download from our Google Code page (\url{http://code.google.com/p/vapp/}),
which is linked from our project web site.