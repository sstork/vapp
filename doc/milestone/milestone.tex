\documentclass{article}

\usepackage[margin=1in]{geometry}

% Try to make it fit on one page
\addtolength{\topmargin}{-.5in}
\addtolength{\textheight}{1in}

\title{15-740 Project Milestone: Virtual Application Profiler}
\author{Sven Stork \\ \texttt{svens@cs.cmu.edu}
	\and Anthony Gitter \\ \texttt{agitter@cs.cmu.edu}}

\begin{document}

\maketitle

\vspace{-0.25in}

\section{Status}
We are pleased to report that our project has been proceeding
according to schedule and that we are on track to achieve our
proposed 100\% goals.  We have not had problems working with the
external software required for our project (Pin and SQLite)
and have all resources needed to complete our project.

Our initial 100\% goal was intentionaly open-ended with regard to what
specific analysis we intended to implement.  After meeting with Dr.
Mowry and discussing what would be most interesting,
we have refined our goals.  Specifically, we will no longer emphasize
analysis that can be performed with pure SQL queries
because this feature is largely architecture-independent.  Instead,
we will focus our replay and analysis on thread interactions in a
multicore system.  This will ensure that our profiler and evaluation
are centered on issues that arise in parallel programs on multicore
machines as opposed to profiling application behavior in a manner that
is not tied to the underlying architecture.

The only major surprise we have encountered thus far has been the
size of our SQLite database logs.  We believe this problem to
be manageable because we modified our profiler to
support dynamic trace enabling/disabling
(details below).  As a new 125\% goal, we could explore more
sophisticated solutions that have been employed in related work,
such as Netzer's transitive reduction \cite{netzer1993optimal}.

\section{Accomplishments}
We implemented a Pin tool that can collect information such
as memory accesses, memory allocation, function calls, and virtual time
and log this data in a SQLite database.  Furthermore, basic profiling and
replay functionality is in place and we are able to use SQL queries to
play back the logged data.  As a proof of concept, we have adapted our
cache simulator from the first assignment to work with our
profiler's replay feature.

To address the aforementioned issue of log size, we modified the
profiler so that an application programmer can selectively decide what
information should logged and at which points in the application
it should be logged.  Tracing is enabled
and disabled by a simple function call to our library function.

The work we have completed achieves the 75\% goal we set in our
project proposal, thus we have successfully met our milestone.

\section{Revised Schedule}
Next week, Sven will focus on modifying our profiler
to log and replay additional information needed to analyze multicore
thread interactions, such as thread ids and locking operations.
Anthony will primarily
concentrate on developing the algorithms used for the analysis of
this data.   In the final week, we will both evaluate our system
on test applications, prepare the poster, and write the final
report.

\bibliographystyle{plain}
\bibliography{milestone}

\end{document}