\section{Lessons Learned}
One surprise we encountered relatively early in the project
was the drawbacks of using a SQLite database to store the
application traces.  Our testing showed explosive growth
of the log file size as we scaled our test programs,
with some log files growing to several hundred megabytes
even on moderate length programs.  In addition, as
seen in our evaluation the writing to the database is a
huge bottleneck in the performance of our Pin tool.
While we maintain that may not be a critical problem if the
user intends to amortize the initial cost of creating the
trace over a very large number of simulated replays,
the number of replays required to make the initial instrumented
execution worthwhile is greater than we expected.  This suggests
that the reason previous work has not exposed the logged data to
the user to the degree we do is that doing so is impractical
with regard to log file size and exaggerated slowdown.

Another unexpected challenge we faced during development was
the difficulty in simulating hardware.  As a 125\% goal we
had hoped to simulate more than just the cache, and had considered
using VAPP to demonstrate the effects of using huge page tables
in a system.  However, we were unable to complete this task and
we now understand how daunting it would be to further develop
VAPP to realistically simulate a variety of real architectures and
configurations of those architectures.